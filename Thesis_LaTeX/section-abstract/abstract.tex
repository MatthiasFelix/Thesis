\begin{abstract}
Online recommender systems have become an important help for users of various applications. They help users deal with the information overload often encountered in today's online world and filter out relevant alternatives. Social recommender systems moreover try to use information contained in social networks to enhance the performance of conventional recommendation methods.

This thesis focuses on the collaborative filtering recommendation approach and analyzes its performance in different environments. A close look is taken at the incorporation of social network information into the collaborative filtering algorithm. Further, concepts from network theory are used to analyze if there exist users who influence the taste of others and therefore could be better predictors.
\end{abstract}
