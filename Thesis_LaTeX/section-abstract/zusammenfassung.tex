\begin{zusammenfassung}
Online-Empfehlungsdienste sind in der heutigen Zeit zu wichtigen Helfern f\"ur Benutzer verschiedenster Applikationen geworden. Sie helfen Benutzern, mit der Informationsflut zurecht zu kommen und relevante Alternativen heraus zu filtern. Soziale Empfehlungsdienste versuchen, mit Hilfe von Informationen in sozialen Netzwerken die Empfehlungen herk\"ommlicher Systeme zu verbessern.

Diese Arbeit konzentriert sich auf die Collaborative-Filtering-Methode und untersucht dessen Performance unter verschiedenen Bedingungen. Weiter wird die Einbindung von Information aus sozialen Netzwerken in den Collaborative-Filtering-Algorithmus genauer studiert. Es werden ausserdem Konzepte aus der Netzwerk-Theorie verwendet, um herauszufinden, ob es Benutzer gibt, welche den Geschmack anderer beeinflussen und somit bessere Indikatoren f\"ur Empfehlungen sein k\"onnten.
\end{zusammenfassung}
