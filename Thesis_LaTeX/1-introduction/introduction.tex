\chapter{Introduction}
\label{c:introduction} Recommender systems have become a widely used tool for users of different kinds of online services. They can help to deal with the information overload and find content relevant for them more easily and reliably. There is a variety of different approaches to recommender systems, of which two main categories can be established: content-based filtering and collaborative filtering. The latter tries to recommend items to a user that other users with similar taste have liked before.

At the same time, online social networks are becoming more and more important in our society. There are endless possibilities to connect with new people, share content that you like or discover new content that others propose to you. These facts lead to the idea of using the information contained in social networks in order to enhance the performance of recommendation algorithms. The information contained in a social graph can be used to find users with similar tastes who could be better indicators of a user's preferences. The combination of recommendation algorithms and social network information gave rise to a new class of algorithms called ``social collaborative filtering''. There have been numerous studies on this topic, some of which report better performance with social collaborative filtering algorithms than with conventional ones \cite{Zheng_2008}, \cite{Konstas_2009}, \cite{Liu_2010}.

In this thesis, the focus lies on systematically inspecting and analyzing conventional and social collaborative filtering. Different social collaboravite filtering algorithms will be compared to conventional ones and in addition, graph and network analysis techniques will be used in order to find conditions under which social collaborative filtering might lead to better performance or to find users with special positions in a social networks that could make them better predictors.

The main research questions that will be asked and analyzed in this thesis are:
\begin{itemize}
\item Under which conditions does social collaborative filtering perform better than conventional collaborative filtering?
\item Are there users for whom social collaborative filtering works better than convenctional collaborative filtering?
\item Are there users that are better predictors than others?
\end{itemize}

To answer these questions, computer experiments and simulations will be performed on real world and artificial datasets, with a software simulation environment built specifically for this purpose. As real world dataset, a publicly available dataset from the social music streaming service last.fm will be used. Further, artificial social networks are created using a well-suited network generation algorithm, as well as artificial rating data.